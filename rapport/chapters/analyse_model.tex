\chapter{Analyse des données du modèle}

\paragraph{Paramètres initiaux}
Cette simulation as été réalisée avec 100 réitérations pour chaque combinaison possible avec les paramètres suivants.

\img{./data/model_parameters}

\section{Analyse d'ensemble}


\img{./data/model_overview_short}
\paragraph{Un seul facteur sort du lot}
À commencer par le facteur C qui est le seul avec semble vraiment avoir de nette relation avec le pourcentage de victoires.

Les autres facteurs ne semblent pas particulièrement affecter les chances de victoires en général pour le moment.


\section{Analyses détaillées}


\subsection{Découpage du spectre selon la taille de la carte}
\img{./data/model_area}
\paragraph{Aucun impact réel décelé}
La taille de la carte ne semble pas influer sur les pourcentages de victoire.




\subsection{Découpage du spectre selon des intervalles de C}
\img{./data/model_C}
\paragraph{C est définitivement un facteur MAJEUR}
Cette représentation montre bien l'influence de C sur le pourcentage de victoire, aussitôt $C \ge 3$ atteinte, les chances de victoires enregistrées sont au minimum de 60-80\%.

Ceci dit, les chances de victoires semblent stagner vers 75-90\% à des valeurs plus élevées. Ce qui reste tout de même majoritairement favoriser la coalition.



\subsection{Découpage du spectre selon des intervalles de Ds}
\img{./data/model_Ds}
\paragraph{Une légère diminution des chances de victoires}
Ds ne semble vraiment pas influer sur les différents spectres non plus, cependant, l'écart de probabilité se creuse avec des valeurs plus grande de Ds, et semble accentuer les tendances de C.

À plus grandes valeurs de Ds, les probabilités de vitoires semblent plus nuancées et défavoriser la coalition.


\subsection{Découpage du spectre selon des intervalles de Dc}
\img{./data/model_Dc}
\paragraph{Aucun impact réel décelé}
Dc ne semble vraiment pas influer sur les différents spectres, par conséquent, Dc ne semble pas avoir d'impact tout court sur notre distribution de probabilité.



\subsection{Découpage du spectre selon des intervalles de la différence de niveau}
\img{./data/model_Dc-Ds}
\paragraph{Toujours aucun impact réel décelé}
La différence d'intelligence ne semble pas non plus influer particulièrement sur les chances de victoires. Il ne semble donc pas y avoir de véritable interaction entre les équipes.



\subsection{Conclusions d'analyse}


\paragraph{Le contrôle de la carte est le facteur le plus important d'après notre modèle}
La seule véritable corrélation que nous avons pu déceler ici est en fonction du nombre de joueurs présents sur la carte.

Ce modèle ne colle malheureusement pas au comportement complet de notre intelligence artificielle, mais permet d'approximer à moindres coûts une stratégie de survie pure, comme lorsqu'un joueur est bloqué dans une zone sans adversaires.

\begin{result}
	Avec une stratégie purement défensive (d'esquive de murs), et un unique joueur dans son équipe, un joueur seul as plus de chance d'éliminer son équipe sur la durée qu'une équipe nombreuse jouant aussi en pure défensive sur plus de surface.
\end{result}

\img{./data/model_conclu}
