\chapter{Analyse des données de l'IA}

\paragraph{Paramètres initiaux}
Cette simulation as été réalisée avec 100 réitérations pour chaque combinaison possible avec les paramètres suivants.
\img{./data/ai_parameters}

\begin{info}
	On remarquera que la profondeur de recherche as une petite fourchette dans notre espace de recherche pour des raisons de temps de calcul. Une analyse plus poussée dans de petites cartes pourrait être intéressante.
\end{info}

\section{Analyse d'ensemble}
\img{./data/ai_overview_short}
\paragraph{Des tendances variées}
Nous pouvons d'abord remarquer la présence de cinq tendances générales de données:
\begin{itemize}
	\item Plus la taille de la map est grande, plus les chances de victoires diminuent.
	\item Plus la taille de la coalition augmente, plus les chances de victoires semblent augmenter en général.
	\item Plus Ds augmente, et plus le chances minimales de victoire diminuent.
	\item Inversement pour Dc, celles-ci augmentent avec Dc.
	\item Plus la différence d'intelligence entre le joueur solo et la coalition augmente, plus les chances générales de victoires diminuent.
\end{itemize}



\section{Analyses détaillées}
\subsection{Répartitions des pourcentages de victoires}
\img{./data/ai_victory_percent}
\paragraph{L'écart d'intelligence semble être primordial}
Les victoires les plus assurées semblent être principalement réparties sur des couples de paramètres où la coalition et le joueur solo ont des intelligences proches, ainsi que lorsque Ds et Dc sont plus élevés.
Les risques de défaites semblent être limités par de plus grandes valeurs de C et Dc, et de faibles valeurs de Ds.
Les cartes plus petites semblent aussi être un avantage pour la coalition.


\subsection{Découpage du spectre selon des intervalles de la taille de la carte}
\img{./data/ai_area}
\paragraph{Petites cartes avantageuses, grandes cartes handicapantes}
Nous pouvons remarquer que sur les petites cartes, les chances maximales de victoires sont exacerbées sur tous les facteurs comparé aux autres tailles.

En revanche, les grandes cartes semblent avoir un effet négatif sur les chances minimales de victoire sur tous les facteurs. 

\subsection{Découpage du spectre selon des intervalles de C}
\img{./data/ai_C}
\paragraph{Le surnombre semble avantageux}
Les chances de victoire générales semblent augmenter avec de plus grandes valeurs de C de façon homogène.


\subsection{Découpage du spectre selon des intervalles de Ds}
\img{./data/ai_Ds}
\paragraph{Un joueur solo intelligent est plus difficile à battre}
Nous pouvons remarquer que de plus grandes valeurs de Ds semblent diminuer les chances générales de victoires.

\subsection{Découpage du spectre selon des intervalles de Dc}
\img{./data/ai_Dc}
\paragraph{Une coalition intelligente bats le joueur plus facilement}
Inversement, les plus grandes valeurs de Dc semblent maximiser les chances générales de victoire.


\subsection{Découpage du spectre selon des intervalles de la différence de niveau}
\img{./data/ai_Dc-Ds}
\paragraph{L'écart d'intelligence est primordial}
Nous pouvons clairement constater que l'écart d'intelligence diminue fortement les chances générales de victoires de façon drastique, passant de 30-90\% pour $diffD = 1$ à 20-55\% pour $diffD = 4$.

Nous pouvons aussi constater une légère augmentation de ces chances de victoires quand Dc et Ds sont maximisées, ce qui pourrait indiquer que l'écart relatif ($diffD = Ds/Dc$) d'intelligence pourrait être une mesure plus précise que l'écart absolu que nous utilisons ici ($diffD = Ds - Dc$).

\subsection{Conclusions d'analyse}

\paragraph{À niveau égal, rien ne va plus!}
D'après nos résultats, nous pouvons constater que plus la différence relative d'intelligence est petite, plus la coalition as de chances d'écraser le joueur solo.

De plus, et ce surtout à partir d'un écart d'intelligence assez important, la taille de la carte semble indiquer que les cartes petites semblent favoriser la coalition, et que les grandes cartes semblent favoriser fortement le joueur solo. 

En revanche, le surnombre semble être un avantage en toutes circonstances, mais son effet semble plaffonner plus vite si l ecart d intelligence est moindre.


\begin{result}
	La pire coalition possible serait une coalition qui joue quasiment au hasard, avec peu d'équipiers et sur une grande carte.
	
	En revanche, la coalition idéale semble être une coalition en surnombre, d'intelligence égale au joueur solo, et sur une petite carte. Mais même juste à niveau égal, sur toute carte moyenne ou petite, la coalition est statistiquement meilleure. 
	
	C'est une bataille de cerveau.
\end{result}

\img{./data/ai_conclu}