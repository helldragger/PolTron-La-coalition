\chapter{Analyse des données de l'IA}

\paragraph{Paramètres initiaux}
Cette simulation as été réalisée avec 100 réitérations pour chaque combinaison possible avec les paramètres suivants.
\img{./data/ai_parameters}

\begin{info}
	On remarquera que la profondeur de recherche as une petite fourchette dans notre espace de recherche pour des raisons de temps de calcul. Une analyse plus poussée dans de petites cartes pourrait être intéressante.
\end{info}

\section{Analyse d'ensemble}
\img{./data/ai_overview_short}
\paragraph{Des données très localisées}
Nous pouvons d'abord remarquer la présence de trois groupes de données bien distincts:
\begin{itemize}
	\item Un groupe de paramètres à la victoire quasiment garantie
	\item Un groupe de paramètres à la défaite quasiment garantie
	\item Un groupe de paramètres plus étalé mais avec de faibles chances de victoire
\end{itemize}

Les facteurs M,N, C et area semble n'avoir aucune influence sur le pourcentage de victoire.

En revanche, il est intéressant de noter que Dc, Ds, et leur différence semblent fortement influer sur ce pourcentage.


\img{./data/ai_overview_complete}

\section{Analyses détaillées}
\subsection{Répartitions des pourcentages de victoires}
\img{./data/ai_victory_percent}
\paragraph{La différence d'intelligence semble être primordiale}
Les victoires semble être principalement réparties sur des couples de paramètres où la coalition et le joueur solo sont aussi intelligents, ainsi que lorsque Ds et Dc sont plus élevés.

La répartition en M, N, C et area est homogène parmi les pourcentages de victoire, donc ne semblent pas influer sur le résultat d'une partie.


\subsection{Découpage du spectre selon des intervalles de C}
\img{./data/ai_C}
\paragraph{Aucune variation particulière}
Malgré le manque de données empêchant la détermination de courbes, l'aspect général des spectres ne varie pas selon C.


\subsection{Découpage du spectre selon des intervalles de Dc}
\img{./data/ai_Dc}
\paragraph{Une scission importante}
Dc influe clairement sur les pourcentages de victoire, sur notre set de données, une haute valeur de Dc semble garantir la victoire. 


\subsection{Découpage du spectre selon des intervalles de Ds}
\img{./data/ai_Ds}
\paragraph{Une autre scission importante}
Ds semble influer aussi beaucoup sur le résultat d'une partie, une faible valeur de Ds semble garantir la défaite de la coalition, tandis que des valeurs plus grandes semblent faire émerger des groupes selon la différence d'intelligence.


\subsection{Découpage du spectre selon des intervalles de l'aire M*N}
\img{./data/ai_area}
\paragraph{Aucune variation particulière}
Nous ne pouvons pas constater de variation particulière des spectres en fonction de l'aire du plateau. Celui ci ne semble donc pas influer sur les chances de victoire.

\subsection{Découpage du spectre selon des intervalles de la différence de niveau}
\img{./data/ai_Dc-Ds}
\paragraph{Des groupes intéressants}
À faible différence d'intelligence, nous pouvons voir que M,N,C et area semblent parfaitement équilibrés, et la différence soudaine de chances de victoires entre de faibles valeurs de Ds, Dc, et de plus grandes valeurs de Ds, Dc.

En revanche à plus grande différence d'intelligence, les chances de victoires semblent être plus constantes et plutôt faibles, sur toutes les combinaisons de valeurs. 

\subsection{Conclusions d'analyse}

\paragraph{À niveau égal, rien ne va plus!}
D'après nos résultats, nous pouvons constater que plus la différence relative d'intelligence est petite, plus la coalition as de chances d'écraser le joueur solo, et ce indépendamment du nombre de joueurs ou de la taille de la carte.

Cependant, notre fourchette de donnée est limitée concernant les différentes valeurs d'intelligence, et une analyse plus poussée sur ces variables là en particulier, suite à des optimisations nécessaires, pourrait permettre d'y voir plus clair. 

\begin{result}
	 À intelligence proche, la coalition semble invincible.
\end{result}

\img{./data/ai_conclu}