\chapter{Analyse - Exploration}

	\section{Nécessités}
	
		\paragraph{Déterminer les facteurs d'une victoire}
		L'objectif final de ce projet est de déterminer les meilleurs paramètres initiaux permettant de maximiser le taux de victoires de la coalition.
		
		Cela étant dit, notre objectif pour y parvenir est d'utiliser l'outil de l'analyse statistique, mais sur les données d'un demi-million de parties potentielles avec une demi douzaine de facteurs différents, par où commencer?
 
	
	\section{Problème}
		
		\paragraph{Comment déterminer les bonnes corrélations?}
		Faire des statistiques à partir de l'intégralité de nos données permet de déterminer des tendances générales, mais sur notre cas où nous avons 5 dimensions de données indépendantes, et donc potentiellement des variations de ces corrélations à chaque modification infime de n'importe quel facteur, comment pourrions nous analyser relativement efficacement l'évolution de ces tendances pour tenter de déterminer de potentielles corrélations cachées entre plusieurs variables?
		
		\img{./pics/anal_partial.png}
	
		\begin{problem}
			Comment pourrions-nous analyser nos données pour pouvoir y déceler des informations de la façon la plus efficace et complète possible sur autant d'axes?
		\end{problem}
	
	\section{Approche utilisée}
	
		\paragraph{Analyse par tranches}
		La quantité inconnue de données que nous avons pour chaque variable, construire une matrice à 5 dimensions serait prohibitif pour nos moyens actuels, autant en espace mémoire qu'en temps de calcul, d'analyse et de génération.
		C'est pour cette raison que nous avons opté pour une simple analyse par intervalles de données présentes.
	
		\begin{result}
			Analyser les tendances de victoire en fixant une variable ou plusieurs variables à la fois et en scindant nos données en 5 intervalles de tailles équivalentes nous permet d'analyser la progression des spectres entre de grandes variations des variables en questions et d'avoir une idée générale des relations entre variables.
		\end{result}
	
		\img{./pics/anal_tranches}
		
	\section{Remarques sur les résultats obtenus}
	
		\paragraph{Les données sont parlantes}
		Voir progresser les intervalles de données disponibles en fonction des intervalles de chaque variable et les voir se chevaucher petit à petit permet vraiment d'avoir une meilleure idée de ce que représente chaque intervalle dans la totalité des données présentes.
		De plus, la comparaison aisée entre les différents spectres à différents intervalles montrent bel et bien si les spectres changent beaucoup ou non selon tel ou tel intervalle d'une variable et permet de déterminer l'influence de cette variable sur ces spectres ou non.
		
	\section{Pistes d'amélioration}
	
		\paragraph{Génération d'un profil 5D voire n-D de probabilités!}
		Comme dit plus haut, à partir de ce genre de données il paraitraît tout naturel de tenter de modéliser un spectre 5D de probabilités permettant de déterminer automatiquement la probabilité de victoire d'un couple de paramètres initiaux arbitraires.
		
		Cela pourrait d'ailleurs être un sujet bien chargé très intéressant à implémenter et à travailler qui pourrait permettre de s'intéresser à des sujets peu communs comme les tenseurs et l'interpolation!
		